\documentclass[numbers=noenddot, 12pt, a4paper, oneside]{scrbook}
\usepackage{blindtext}
\usepackage[utf8]{inputenc}
\usepackage{float}
\usepackage{tabularx}
\usepackage{graphicx}
\def\Plus{\texttt{+}}
\usepackage{listings}
\usepackage{color}

\definecolor{dkgreen}{rgb}{0,0.6,0}
\definecolor{gray}{rgb}{0.5,0.5,0.5}
\definecolor{mauve}{rgb}{0.58,0,0.82}

\lstset{frame=tb,
	language=Java,
	aboveskip=3mm,
	belowskip=3mm,
	showstringspaces=false,
	columns=flexible,
	basicstyle={\small\ttfamily},
	numbers=none,
	numberstyle=\tiny\color{gray},
	keywordstyle=\color{blue},
	commentstyle=\color{dkgreen},
	stringstyle=\color{mauve},
	breaklines=false,
	breakatwhitespace=true,
	tabsize=3
}


\begin{document}

\begin{titlepage}
	\centering
	{\scshape\LARGE Politecnico di Milano \par}
	\vspace{1cm}
	\includegraphics[width=0.35\textwidth]{polimi-logo}\par
	\vspace{1cm}

	{\scshape\Large Design and Implementation of Mobile Applications\par}
	\vspace{1.5cm}
	{\huge\bfseries iSport \par}
	\vspace{1cm}
	{\Large\bfseries Design Document \par}
	\vspace{3cm}
	{\Large\itshape di\par}
	{\Large\itshape Gianluigi Oliva\par}
	\vspace{1.5cm}
	\vfill
	


	\vfill

	% Bottom of the page
	{\large \today\par}
\end{titlepage}

\newpage
\tableofcontents
\newpage


\chapter{Introduction}

\section{Purpose}
Questo documento descrive le fasi di progettazione e prototipazione per la realizzazione dell'applicazione mobile "iSport". in dettaglio, verranno discussi i componenti principali, le funzionalità e l'esperienza dell'utente.

iSport è una applicazione in cui obiettivo principale è la visualizzazione di informazioni e dati relativi al mondo dello sport. In particolare ci si focalizzerà sulle notizie giornalistiche più rilevanti e sui dati relativi alle partite di calcio della giornata.

Questo progetto è il risultato dell'implementazione delle conoscenze acquisite durante il corso "Design and Implementation of Mobile Applications" fornito dal Politecnico di Milano.

\section{Intended Audience}
This document is produced for those who develop, evaluate and use iSport mobile application:
\begin{itemize}
	\item The engineers who had the idea and developed the application.
	\item The testers that must verify the effective implementation of all the described components and functions.
	\item The user who will use the application and take advantage of its functionalities.
	\item The future contributors who wish to develop new features.
\end{itemize}

\section{Definitions, acronyms, abbreviations}
\subsection*{Definitions}
\begin{itemize}
	\item \textbf{Platform}: The application as a whole.
	\item \textbf{User}: An end user who will use the application
	\item \textbf{Match}: A match between two teams that has already occurred or is in progress
	\item \textbf{Framework}: Reusable set of libraries or classes for a software system.
	\item \textbf{News}: Una notizia relativa al mondo dello sport presente su qualche rivista giornalistica
	\item \textbf{Pronostico}: Una previsione sul risultato di un match appartenente ad una classe di previsioni possibili
	\item \textbf{Quota}: Il valore di retribuzione di un pronostico relativo ad un determinato match
	\item \textbf{REST}: is a way of providing interoperability between computer systems on the Internet.
\end{itemize}
\subsection*{Acronyms}
\begin{itemize}
	\item \textbf{MVC}: Model - View - Controller
	\item \textbf{HTTPS}: HyperText Transfer Protocol Secure
	\item \textbf{IDE}: Integrated Development Environment
	\item \textbf{API}: Application Programming Interface
	\item \textbf{JSON}: JavaScript Object Notation
	\item \textbf{UML}: Unified Modelling Language.
	\item \textbf{UX}: User Experience
	\item \textbf{URL}: Uniform Resource Locator
\end{itemize}
\subsection*{Abbreviations}
\begin{itemize}
	\item \textbf{App}: Mobile Application 
\end{itemize}

\section{Mobile Application Scope}
iSport è stato sviluppato per tutti coloro che sono appassionati di sport cercando di unificare sotto un'unica applicazione tutti i servizi presenti sul mercato. In questo modo si vuole dare più continuità di utilizzo all'utente finale, senza che egli abbia la necessità di navigare su più applicazioni per ottenere lo stesso risultato.\\
In particolare l'applicazione si articolerà in tre schermate:
\begin{itemize}
	\item \textbf{News}
	\item \textbf{Live}
	\item \textbf{Bet}
\end{itemize}
Nella sezione "News" saranno presenti le principali notizie relative al mondo dello sport visualizzabili con un'immagine di anteprima e una piccola descrizione. Inoltre premendo sulla singola notizia sarà possibile visualizzare l'articolo completo.\\
Nella sezione "Live" saranno presenti tutte le partite della giornata corrente con il risultato della partita se già conclusa o quello attuale se ancora in corso. Premendo sulla singola partita sarà possibile consultare tutte le informazioni su di essa come i marcatori e il minuto del goal, cartellini, formazione e statistiche.\\
Nella sezione "Bet" saranno presenti le quote relative ai principali pronostici delle partite odierne. Premendo sulle singole quote sarà possibile comporre una schedina e una volta conclusa l'applicazione calcolerà la potenziale vincita in base all'importo della giocata.

\section{Framework}
Lo sviluppo di iSport è stato realizzato mediante l'utilizzo delle native SDK iOS, in particolare facendo ricorso al linguaggio di programmazione Swift. Questa scelta è stata fatta per permettere un maggior controllo delle risorse del sistema e l'accesso a dei servizi di sistema non possibili con l'utilizzo di framework cross-platform come PhoneGap o React Native. I principali obiettivi sono stati l'implementazione di diverse funzionalità e l'integrazione con altri siti.


\chapter{General Overview}
\section{Core Features}
\section{Functional Requirements}
\section{Non Functional Requirements}

\chapter{Data Design}
\section{E/R Model}
\section{Core Data}

\chapter{Architecture}
\section{General Overview}
\section{Server}
\section{Client}
\section{Patterns}
\section{Security}

\chapter{User Interfaces}


\chapter{External Services and Libraries}

\chapter{UML}

\chapter{Test Cases}
\chapter{Cost Estimation}







\end{document}
